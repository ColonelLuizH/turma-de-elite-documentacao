%% Adaptado a partir de :
%%    abtex2-modelo-trabalho-academico.tex, v-1.9.2 laurocesar
%% para ser um modelo para os trabalhos no IFSP-SPO

\documentclass[
    % -- opções da classe memoir --
    12pt,               % tamanho da fonte
    openright,          % capítulos começam em pág ímpar (insere página vazia caso preciso)
    %twoside,            % para impressão em verso e anverso. Oposto a oneside
    oneside,
    a4paper,            % tamanho do papel. 
    % -- opções da classe abntex2 --schwinn
    % Opções que não devem ser utilizadas na versão final do documento
    %draft,              % para compilar mais rápido, remover na versão final
    % paginasA3,  % indica que vai utilizar paginas em A3 
    % MODELO,             % indica que é um documento modelo então precisa dos geradores de texto
    % TODO,               % indica que deve apresentar lista de pendencias 
    % -- opções do pacote babel --
    english,            % idioma adicional para hifenização
    brazil              % o último idioma é o principal do documento
    ]{ifsp-spo-inf-ctds} % ajustar de acordo com o modelo desejado para o curso

% ---
% Pacotes importados para a utilização de referências
% ---

% ---
% Informações de dados para CAPA e FOLHA DE ROSTO
% ---
\titulo{AVALIAÇÃO DAS OUTRAS EQUIPES - DESENHO DA APLICAÇÃO}

% Trabalho individual
%\autor{AUTOR DO TRABALHO}

% Trabalho em Equipe
% ver também https://github.com/abntex/abntex2/wiki/FAQ#como-adicionar-mais-de-um-autor-ao-meu-projeto
\renewcommand{\imprimirautor}{
\begin{tabular}{lr}
     André Monteiro GOMES & SP3024059 \\
     Bianca Kaori HNG & SP3022455\\
     Luiz Henrique de Almeida e ALBUQUERQUE & SP3030199\\
     Natan da Fonseca LISBOA & SP3024784\\
     Patrícia Santos PASCHOAL & SP3022218
\end{tabular}
}


\disciplina{PI1A5 - Projeto Integrado I}

\preambulo{Trabalho apresentado ao Instituto Federal de Educação, Ciência e Tecnologia de São Paulo - Câmpus São Paulo - como parte dos requisitos para aprovação na disciplina Projeto Integrado I (PI1A5), do curso superior de Tecnologia em Análise e Desenvolvimento de Sistemas.}

\data{2021}

% Definir o que for necessário e comentar o que não for necessário
% Utilizar o Nome Completo, abntex tem orientador e coorientador
% então vão ser utilizados na definição de professor
\renewcommand{\orientadorname}{Professor:}
\orientador{DANIEL MARQUES GOMES DE MORAIS}


% ---


% informações do PDF
\makeatletter
\hypersetup{
        %pagebackref=true,
        pdftitle={\@title}, 
        pdfauthor={\@author},
        pdfsubject={\imprimirpreambulo},
        pdfcreator={LaTeX with abnTeX2 using IFSP model},
        pdfkeywords={abnt}{latex}{abntex}{abntex2}{IFSP}{\ifspprefixo}{trabalho acadêmico}, 
        colorlinks=true,            % false: boxed links; true: colored links
        linkcolor=blue,             % color of internal links
        citecolor=blue,             % color of links to bibliography
        filecolor=magenta,              % color of file links
        urlcolor=blue,
        bookmarksdepth=4
}
\makeatother
% --- 

% ----
% Início do documento
% ----
\begin{document}

\imprimircapa

\pdfbookmark[0]{\contentsname}{toc}
\tableofcontents*

\textual

\chapter[Equipe AcadTech]{Equipe AcadTech}
A equipe AcadTech tem como projeto a aplicação MovieStar.


\section{Projeto MovieStar}
O projeto MovieStar tem como intuito a condensação de vários resumos de obras cinematográficas na plataforma, a fim de que os usuários possam avaliá-las e discuti-las.


Visto que o projeto será executado em ambientes Web, a arquitetura do projeto está coerente. Entretanto, vale ressaltar alguns detalhes de implementação que podem ser considerados. É cabível ressaltar também que o problema inicial do sistema gerenciador de banco de dados escolhido foi resolvido, uma vez que a equipe passará a utilizar a solução de código aberto PostgreSQL para armazenamento e gerenciamento das informações, não mais o SQL Server.


Apesar da ideia de algo parecido com uma rede social ter sido apresentada inicialmente, o que é pouco possível de ser executada no contexto da disciplina, a equipe conseguiu diminuir a dependência da plataforma com relação ao seu módulo de comunicação entre usuários, focando nas funcionalidades de avaliação das obras cinematográficas.


Com relação a apresentação visual, há apenas alguns detalhes que precisam ser revistos. Entre eles, está o desenho da arquitetura geral do sistema, no qual não foi possível distinguir com clareza a comunicação entre as camadas da aplicação e as tecnologias que as compõem.


Por fim, cabe enfatizar que a viabilidade financeira para o projeto foi definida, com a adoção de um plano Premium a um preço fixo que garante maiores funcionalidades aos usuários que acessarem a plataforma.


\subsection{Sugestões de melhorias}
Para a apresentação visual, uma solução plausível para a representação da arquitetura do sistema seria a nomeação de cada parte da aplicação em cada círculo (com \textit{back-end} e \textit{front-end}, por exemplo), para que a representação das partes do sistema, com as suas respectivas tecnologias, fique mais clara. 


Para a arquitetura do projeto, uma sugestão simples e intuitiva que pode ser implementada é a maior separação das classes no modelo MVC por meio da criação de pacotes de serviço para realizar a interação com o banco de dados, e pacotes de exibição, para prevenir a apresentação de dados sensíveis ao usuário, como identificadores pessoais e senhas, por exemplo.
\chapter[Equipe ConsacreTADS]{Equipe ConsacreTADS}
A equipe ConsacreTADS tem como projeto o aplicativo Safira.


\section{Projeto Safira}
O projeto Safira é um aplicativo para auxiliar as pessoas a organizarem suas vidas financeiras. Nesse aplicativo, os usuários poderão cadastrar suas entradas e saídas, estabelecer metas de gastos e categorizar suas despesas, bem como visualizar gráficos de seus gastos.


A apresentação foi bem distribuída e seguiu o tempo estipulado, entretanto ficou faltando elementos visuais para prender a atenção do público. 


As tecnologias foram melhor definidas e a monetização foi alterada de modo a possuir um plano Premium para assinantes que quiserem mais funcionalidades.


A arquitetura da aplicação foi definida, entretanto faltaram alguns entregáveis de modelagem, como protótipos de baixa fidelidade e a modelagem de dados.


\subsection{Sugestões de melhorias}
Uma melhoria seria desenhar os protótipos e fazer a modelagem de dados da aplicação. Além disso, seria interessante detalhar melhor o plano de testes, até mesmo para facilitar na hora do seu desenvolvimento.

E uma dica seria estudar as apresentações dos projetos anteriores ou até mesmo procurar em sites de pesquisa, para poder ter uma base de como explorar melhor os recursos visuais em apresentações.


\end{document}