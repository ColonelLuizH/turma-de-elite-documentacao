\chapter{Equipe ConsacreTADS}

A equipe ConsacreTADS tem como projeto a aplicação Safira.

\section{Validação da arquitetura da aplicação}
A arquitetura da aplicação Safira contempla desenvolvimento \textit{\gls{front-end}} com o \textit{\gls{framework}} React Native, amplamente utilizado para o desenvolvimento de soluções para dispositivos móveis.


Para o \textit{\gls{back-end}}, está sendo utilizado \textit{Node.js} para desenvolvimento Javascript, além de PostgreSQL para gerenciamento do banco de dados relacional.


Em linhas gerais, a prova de conceito da aplicação foi validada com uma apresentação concisa, na qual foi demonstrada a comunicação entre o \textit{\gls{front-end}} e o \textit{\gls{back-end}} por meio do login de um usuário no sistema. Entretanto, o \textit{\gls{deploy}} da aplicação não foi demonstrado.

\section{Sugestões de melhorias}
Uma solução em alternativa ao \textit{\gls{deploy}} da aplicação é a integração contínua que pode ser implementada por meio do \textit{GitHub Actions}, uma ferramenta fornecida pelo próprio \textit{\gls{github}} para gerenciamento de intregração do desenvolvimento, teste e disponibilização do código realizado.


Para substituir as \textit{\glspl{query}} manuais demonstradas para busca de um usuário no sistema, é cabível a utilização de uma ferramenta para geração de \textit{\glspl{query}} automatizadas, como o Sequelize, por exemplo.


Por fim, a internacionalização da aplicação pode ser feita por um \textit{\gls{framework}} disponível ao React Native, o \textit{\gls{react-i18next}}.