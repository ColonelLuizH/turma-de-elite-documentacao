%% Adaptado a partir de :
%%    abtex2-modelo-trabalho-academico.tex, v-1.9.2 laurocesar
%% para ser um modelo para os trabalhos no IFSP-SPO

\documentclass[
    % -- opções da classe memoir --
    12pt,               % tamanho da fonte
    openright,          % capítulos começam em pág ímpar (insere página vazia caso preciso)
    %twoside,            % para impressão em verso e anverso. Oposto a oneside
    oneside,
    a4paper,            % tamanho do papel. 
    % -- opções da classe abntex2 --schwinn
    % Opções que não devem ser utilizadas na versão final do documento
    %draft,              % para compilar mais rápido, remover na versão final
    % paginasA3,  % indica que vai utilizar paginas em A3 
    % MODELO,             % indica que é um documento modelo então precisa dos geradores de texto
    % TODO,               % indica que deve apresentar lista de pendencias 
    % -- opções do pacote babel --
    english,            % idioma adicional para hifenização
    brazil              % o último idioma é o principal do documento
    ]{ifsp-spo-inf-ctds} % ajustar de acordo com o modelo desejado para o curso

% ---
% Pacotes importados para a utilização de referências
% ---

% ---
% Informações de dados para CAPA e FOLHA DE ROSTO
% ---
\titulo{AVALIAÇÃO DAS OUTRAS EQUIPES - PROVA DE CONCEITO}

% Trabalho individual
%\autor{AUTOR DO TRABALHO}

% Trabalho em Equipe
% ver também https://github.com/abntex/abntex2/wiki/FAQ#como-adicionar-mais-de-um-autor-ao-meu-projeto
\renewcommand{\imprimirautor}{
\begin{tabular}{lr}
     André Monteiro GOMES & SP3024059 \\
     Bianca Kaori HNG & SP3022455\\
     Luiz Henrique de Almeida e ALBUQUERQUE & SP3030199\\
     Natan da Fonseca LISBOA & SP3024784\\
     Patrícia Santos PASCHOAL & SP3022218
\end{tabular}
}


\disciplina{PI1A5 - Projeto Integrado I}

\preambulo{Trabalho apresentado ao Instituto Federal de Educação, Ciência e Tecnologia de São Paulo - Câmpus São Paulo - como parte dos requisitos para aprovação na disciplina Projeto Integrado I (PI1A5), do curso superior de Tecnologia em Análise e Desenvolvimento de Sistemas.}

\data{2021}

% Definir o que for necessário e comentar o que não for necessário
% Utilizar o Nome Completo, abntex tem orientador e coorientador
% então vão ser utilizados na definição de professor
\renewcommand{\orientadorname}{Professor:}
\orientador{DANIEL MARQUES GOMES DE MORAIS}


% ---


% informações do PDF
\makeatletter
\hypersetup{
        %pagebackref=true,
        pdftitle={\@title}, 
        pdfauthor={\@author},
        pdfsubject={\imprimirpreambulo},
        pdfcreator={LaTeX with abnTeX2 using IFSP model},
        pdfkeywords={abnt}{latex}{abntex}{abntex2}{IFSP}{\ifspprefixo}{trabalho acadêmico}, 
        colorlinks=true,            % false: boxed links; true: colored links
        linkcolor=blue,             % color of internal links
        citecolor=blue,             % color of links to bibliography
        filecolor=magenta,              % color of file links
        urlcolor=blue,
        bookmarksdepth=4
}
\makeatother
% --- 

% ----
% Início do documento
% ----
\begin{document}

\pretextual

\imprimircapa

\pdfbookmark[0]{\contentsname}{toc}
\tableofcontents*

\textual

% ---
% inserir lista de abreviaturas e siglas
% ATENCAO o SHARELATEX/OVERLEAF GERA O GLOSSARIO SOMENTE UMA VEZ
% CASO SEJA FEITA ALGUMA ALTERAÇÃO NA LISTA DE SIGLAS É NECESSARIO UTILIZAR A OPÇÃO :
% "Clear Cached Files" DISPONIVEL NA VISUALIZAÇÃO DOS LOGS 
% ---
% https://www.sharelatex.com/learn/Glossaries


% \ifdef{\printnoidxglossary}{
    % \printnoidxglossary[type=\acronymtype,title=Li% sta de abreviaturas e siglas,style=siglas]
    % \cleardoublepage
% }{}


\chapter{Equipe AcadTech}

A equipe AcadTech tem como projeto a aplicação MovieStar.

\section{Validação da arquitetura da aplicação}
A arquitetura da aplicação MovieStar proposta no desenho do projeto contempla a biblioteca \textit{\gls{react}} para desenvolvimento do \textit{\gls{front-end}}, \textit{\gls{bootstrap}} para fornecer maiores alternativas para customização, além das tecnologias padrão para a construção do \textit{\gls{front-end}}, como HTML, CSS e Javascript.


O \textit{\gls{back-end}}, por sua vez, é composto pelo \textit{\gls{node.js}} para desenvolvimento Javascript do lado do servidor e \textit{PostgreSQL} como banco de dados. Para o \textit{\gls{deploy}} da aplicação, será utilizado o \textit{Amazon EC2}.


O \textit{\gls{front-end}} da aplicação está bem atraente, principalmente com a listagem de filmes a serem escolhidos e a descrição deles quando algum é selecionado. Entretanto, na apresentação não foi demonstrada interação com o \textit{\gls{back-end}} de produção, mas foi utilizada uma \textit{\ac{api}} do Firebase para extração dos dados, o que prejudicou a validação da prova de conceito da aplicação.


Além disso, alguns dos integrantes da equipe não puderam comparecer na apresentação, que foi adiada para a semana seguinte por conta do ocorrido.

\section{Sugestões de melhorias}
Como o \textit{\gls{back-end}} de produção não foi utilizado, não foi possível avaliar a prova de conceito por completo para a sugestão de melhorias estruturais.


Portanto, uma sugestão possível para a próxima apresentação é a utilização do \textit{\gls{back-end}} de produção em substituição à \textit{\ac{api}} utilizada anteriormente.
\chapter{Equipe ConsacreTADS}

A equipe ConsacreTADS tem como projeto a aplicação Safira.

\section{Validação da arquitetura da aplicação}
A arquitetura da aplicação Safira contempla desenvolvimento \textit{\gls{front-end}} com o \textit{\gls{framework}} React Native, amplamente utilizado para o desenvolvimento de soluções para dispositivos móveis.


Para o \textit{\gls{back-end}}, está sendo utilizado \textit{Node.js} para desenvolvimento Javascript, além de PostgreSQL para gerenciamento do banco de dados relacional.


Em linhas gerais, a prova de conceito da aplicação foi validada com uma apresentação concisa, na qual foi demonstrada a comunicação entre o \textit{\gls{front-end}} e o \textit{\gls{back-end}} por meio do login de um usuário no sistema. Entretanto, o \textit{\gls{deploy}} da aplicação não foi demonstrado.

\section{Sugestões de melhorias}
Uma solução em alternativa ao \textit{\gls{deploy}} da aplicação é a integração contínua que pode ser implementada por meio do \textit{GitHub Actions}, uma ferramenta fornecida pelo próprio \textit{\gls{github}} para gerenciamento de intregração do desenvolvimento, teste e disponibilização do código realizado.


Para substituir as \textit{\glspl{query}} manuais demonstradas para busca de um usuário no sistema, é cabível a utilização de uma ferramenta para geração de \textit{\glspl{query}} automatizadas, como o Sequelize, por exemplo.


Por fim, a internacionalização da aplicação pode ser feita por um \textit{\gls{framework}} disponível ao React Native, o \textit{\gls{react-i18next}}.

% ----------------------------------------------------------
% Glossário
% ----------------------------------------------------------
%

 \ifdef{\printnoidxglossary}{
     \addcontentsline{toc}{chapter}{GLOSSÁRIO}
     \printnoidxglossary[style=glossario]
}{}

\end{document}