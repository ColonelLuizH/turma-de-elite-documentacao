\chapter{Equipe AcadTech}

A equipe AcadTech tem como projeto a aplicação MovieStar.

\section{Validação da arquitetura da aplicação}
A arquitetura da aplicação MovieStar proposta no desenho do projeto contempla a biblioteca \textit{\gls{react}} para desenvolvimento do \textit{\gls{front-end}}, \textit{\gls{bootstrap}} para fornecer maiores alternativas para customização, além das tecnologias padrão para a construção do \textit{\gls{front-end}}, como HTML, CSS e Javascript.


O \textit{\gls{back-end}}, por sua vez, é composto pelo \textit{\gls{node.js}} para desenvolvimento Javascript do lado do servidor e \textit{PostgreSQL} como banco de dados. Para o \textit{\gls{deploy}} da aplicação, será utilizado o \textit{Amazon EC2}.


O \textit{\gls{front-end}} da aplicação está bem atraente, principalmente com a listagem de filmes a serem escolhidos e a descrição deles quando algum é selecionado. Entretanto, na apresentação não foi demonstrada interação com o \textit{\gls{back-end}} de produção, mas foi utilizada uma \textit{\ac{api}} do Firebase para extração dos dados, o que prejudicou a validação da prova de conceito da aplicação.


Além disso, alguns dos integrantes da equipe não puderam comparecer na apresentação, que foi adiada para a semana seguinte por conta do ocorrido.

\section{Sugestões de melhorias}
Como o \textit{\gls{back-end}} de produção não foi utilizado, não foi possível avaliar a prova de conceito por completo para a sugestão de melhorias estruturais.


Portanto, uma sugestão possível para a próxima apresentação é a utilização do \textit{\gls{back-end}} de produção em substituição à \textit{\ac{api}} utilizada anteriormente.