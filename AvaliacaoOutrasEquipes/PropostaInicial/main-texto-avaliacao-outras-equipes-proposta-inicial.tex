%% Adaptado a partir de :
%%    abtex2-modelo-trabalho-academico.tex, v-1.9.2 laurocesar
%% para ser um modelo para os trabalhos no IFSP-SPO

\documentclass[
    % -- opções da classe memoir --
    12pt,               % tamanho da fonte
    openright,          % capítulos começam em pág ímpar (insere página vazia caso preciso)
    %twoside,            % para impressão em verso e anverso. Oposto a oneside
    oneside,
    a4paper,            % tamanho do papel. 
    % -- opções da classe abntex2 --schwinn
    % Opções que não devem ser utilizadas na versão final do documento
    %draft,              % para compilar mais rápido, remover na versão final
    % paginasA3,  % indica que vai utilizar paginas em A3 
    % MODELO,             % indica que é um documento modelo então precisa dos geradores de texto
    % TODO,               % indica que deve apresentar lista de pendencias 
    % -- opções do pacote babel --
    english,            % idioma adicional para hifenização
    brazil              % o último idioma é o principal do documento
    ]{ifsp-spo-inf-ctds} % ajustar de acordo com o modelo desejado para o curso

% ---
% Pacotes importados para a utilização de referências
% ---

% ---
% Informações de dados para CAPA e FOLHA DE ROSTO
% ---
\titulo{AVALIAÇÃO DAS OUTRAS EQUIPES - PROPOSTA INICIAL}

% Trabalho individual
%\autor{AUTOR DO TRABALHO}

% Trabalho em Equipe
% ver também https://github.com/abntex/abntex2/wiki/FAQ#como-adicionar-mais-de-um-autor-ao-meu-projeto
\renewcommand{\imprimirautor}{
\begin{tabular}{lr}
     André Monteiro GOMES & SP3024059 \\
     Bianca Kaori HNG & SP3022455\\
     Luiz Henrique de Almeida e ALBUQUERQUE & SP3030199\\
     Natan da Fonseca LISBOA & SP3024784\\
     Patrícia Santos PASCHOAL & SP3022218
\end{tabular}
}


\disciplina{PI1A5 - Projeto Integrado I}

\preambulo{Trabalho apresentado ao Instituto Federal de Educação, Ciência e Tecnologia de São Paulo - Câmpus São Paulo - como parte dos requisitos para aprovação na disciplina Projeto Integrado I (PI1A5), do curso superior de Tecnologia em Análise e Desenvolvimento de Sistemas.}

\data{2021}

% Definir o que for necessário e comentar o que não for necessário
% Utilizar o Nome Completo, abntex tem orientador e coorientador
% então vão ser utilizados na definição de professor
\renewcommand{\orientadorname}{Professor:}
\orientador{DANIEL MARQUES GOMES DE MORAIS}


% ---


% informações do PDF
\makeatletter
\hypersetup{
        %pagebackref=true,
        pdftitle={\@title}, 
        pdfauthor={\@author},
        pdfsubject={\imprimirpreambulo},
        pdfcreator={LaTeX with abnTeX2 using IFSP model},
        pdfkeywords={abnt}{latex}{abntex}{abntex2}{IFSP}{\ifspprefixo}{trabalho acadêmico}, 
        colorlinks=true,            % false: boxed links; true: colored links
        linkcolor=blue,             % color of internal links
        citecolor=blue,             % color of links to bibliography
        filecolor=magenta,              % color of file links
        urlcolor=blue,
        bookmarksdepth=4
}
\makeatother
% --- 

% ----
% Início do documento
% ----
\begin{document}

\pretextual

\imprimircapa

\pdfbookmark[0]{\contentsname}{toc}
\tableofcontents*

\textual

\chapter[Equipe AcadTech]{Equipe AcadTech}
A equipe AcadTech tem como projeto a aplicação MovieStar.

\section{Projeto MovieStar}
Aplicação para avaliar filmes de maneira cooperativa.


\subsection{Proposta}
O projeto visa centralizar as obras cinematográficas em uma plataforma, com o intuito de obter uma interação maior do usuário com elas para que possam avaliá-las em grupo.


\subsection{Características}
O conjunto inicial de tecnologias proposto consiste na utilização da biblioteca React para o desenvolvimento \textit{front-end} e Node.js para o desenvolvimento \textit{back-end}, além do SQL Server para gerenciamento do banco de dados.

    
\subsection{Diferenciais}
Os diferenciais apresentados pela equipe estão concentrados no compartilhamento de atividades de avaliação entre usuários e na interação entre os mesmos, conferindo uma natureza de rede social para a aplicação.


\subsection{Sugestões de melhorias}
Um ponto a ser considerado seria a incapacidade do desenvolvimento de soluções que necessitem de uma grande base de usuários para funcionar, tendo em vista o contexto da disciplina.


Portanto, o ideal seria a equipe focar no processo de avaliação dos filmes, sem que a aplicação dependa de um grande número de acessos simultâneos, como uma rede social necessita.


Além disso, o SQL Server como o sistema gerenciador de banco de dados não é uma boa escolha, visto que ele não possui código aberto e, por isso, pode se tornar custoso de substituí-lo no caso de uma possível mudança de tecnologias. Uma potencial saída para esse problema é a adoção de soluções de código aberto, como o MySQL ou o PostgreSQL, por exemplo.


No aspecto visual da apresentação, uma sugestão de melhoria seria a disposição mais esparsa dos tópicos nas páginas, uma vez que havia muito espaço sobrando por conta da distribuição desproporcional dos pontos abordados, a fim de garantir uma melhor visualização para o leitor.


Por fim, para o gerenciamento do projeto, uma sugestão seria a definição de um plano de testes bem estruturado e um projeto de viabilidade financeira factível, uma vez que tais pontos são essenciais no processo de desenvolvimento da aplicação.
\chapter[Equipe ConsacreTADS]{Equipe ConsacreTADS}
A equipe ConsacreTADS tem como projeto o aplicativo Safira.

\section{Projeto Safira}
Aplicativo para auxiliar as pessoas a organizarem a vida financeira.

\subsection{Proposta}
A proposta é ser uma aplicação, simples e intuitiva, de planejamento financeiro, na qual o usuário pode controlar suas entradas e saídas e organizar seus gastos. 

\subsection{Características}
O aplicativo será desenvolvido usando React Native e Node.js e sua monetização será feita a partir das doações de seus usuários.

\subsection{Diferenciais}
O projeto possui como diferenciais a categorização das despesas, os gráficos de gastos e o estabelecimento de metas, em porcentagem, para os gastos. 

\subsection{Sugestões de melhorias}
Um dos pontos positivos na apresentação foi o fato da cor de fundo contrastar com a cor da letra, entretanto, como melhoria, poderia-se citar a exploração de recursos visuais para ela.

Outro ponto seria a exploração de mais diferenciais para o aplicativo, isto é, algo que incentive as pessoas a usarem a aplicação e não um controle de gastos pessoal.

E, por fim, uma última sugestão seria melhorar a ideia da monetização, uma vez que muitas pessoas não têm consciência do real valor do produto apresentado.

\end{document}