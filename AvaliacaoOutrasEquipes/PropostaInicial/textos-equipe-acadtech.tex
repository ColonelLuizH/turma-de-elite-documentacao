\chapter[Equipe AcadTech]{Equipe AcadTech}
A equipe AcadTech tem como projeto a aplicação MovieStar.

\section{Projeto MovieStar}
Aplicação para avaliar filmes de maneira cooperativa.


\subsection{Proposta}
O projeto visa centralizar as obras cinematográficas em uma plataforma, com o intuito de obter uma interação maior do usuário com elas para que possam avaliá-las em grupo.


\subsection{Características}
O conjunto inicial de tecnologias proposto consiste na utilização da biblioteca React para o desenvolvimento \textit{front-end} e Node.js para o desenvolvimento \textit{back-end}, além do SQL Server para gerenciamento do banco de dados.

    
\subsection{Diferenciais}
Os diferenciais apresentados pela equipe estão concentrados no compartilhamento de atividades de avaliação entre usuários e na interação entre os mesmos, conferindo uma natureza de rede social para a aplicação.


\subsection{Sugestões de melhorias}
Um ponto a ser considerado seria a incapacidade do desenvolvimento de soluções que necessitem de uma grande base de usuários para funcionar, tendo em vista o contexto da disciplina.


Portanto, o ideal seria a equipe focar no processo de avaliação dos filmes, sem que a aplicação dependa de um grande número de acessos simultâneos, como uma rede social necessita.


Além disso, o SQL Server como o sistema gerenciador de banco de dados não é uma boa escolha, visto que ele não possui código aberto e, por isso, pode se tornar custoso de substituí-lo no caso de uma possível mudança de tecnologias. Uma potencial saída para esse problema é a adoção de soluções de código aberto, como o MySQL ou o PostgreSQL, por exemplo.


No aspecto visual da apresentação, uma sugestão de melhoria seria a disposição mais esparsa dos tópicos nas páginas, uma vez que havia muito espaço sobrando por conta da distribuição desproporcional dos pontos abordados, a fim de garantir uma melhor visualização para o leitor.


Por fim, para o gerenciamento do projeto, uma sugestão seria a definição de um plano de testes bem estruturado e um projeto de viabilidade financeira factível, uma vez que tais pontos são essenciais no processo de desenvolvimento da aplicação.