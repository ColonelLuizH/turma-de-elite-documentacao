\chapter{Integrações}
Tendo em vista que já existe a adoção de sistemas gerenciadores de aprendizado por parte das instituições de ensino, foi decidido como um teste de viabilidade futuro a integração da aplicação com um LMS pronto. O LMS teria a responsabilidade de lidar com toda a parte referente à entrega de atividades pelos alunos. O LMS escolhido foi o Moodle, devido aos seguintes motivos:
\begin{itemize}
    \item{ser código aberto;}
    \item{ter a possibilidade de integração com documentação oficial para tal;} 
    \item{alto uso (187000 sites registrados no mundo e 9437 sites registrados no Brasil, segundo o site da plataforma).}
\end{itemize}

A forma de integração testada inicialmente será utilizando Web Services, onde as operações de gravação e leitura de dados na plataforma será feita através de requisições utilizando o protocolo HTTP.