\chapter{Análise comparativa}
Tomando uma plataforma como a Khan Academy, que utiliza conceitos de gamificação para a execução de atividades, nota-se que ela não utiliza o conceito de \textit{tiers} para ranquear os alunos, por exemplo. Esse conceito compreende ao agrupamento de estudantes em ligas diferentes de acordo com o desempenho.

Outro diferencial em relação ao Khan Academy, são as funcionalidades de customização de atividades, parametrização de turmas e conquistas que permitem que a Turma de Elite se adapte não só a diferentes escolas como também as categorias de ensino remoto e à distância.

Uma outra opção possível para gamificar o ensino é a incorporação de um plugin que ofereça ferramentas de gamificação a um sistema de LMS já consolidado no mercado. Ao analisar as funcionalidades de um plugin de gamificação \textit{open-source} chamado "Level Up!", nota-se que ele não contempla algumas funcionalidades para sua utilização no contexto de gestão de uma instituição de ensino, como a visão de gestor, por exemplo.

Além dessas plataformas, foi feita uma análise comparativa com algumas outras. A síntese dos resultados obtidos pode ser representada por meio da tabela abaixo:

\begin{table}[h]

\ABNTEXfontereduzida
\caption{Análise comparativa entre as plataformas de gestão de aprendizado}
\label{tabela-correta-equipamento}
\begin{tabular}[h]{|m{2.0cm}|m{1.5cm}|m{1.5cm}|m{1.5cm}|m{1.5cm}|m{1.5cm}|m{1.5cm}|m{1.5cm}|m{1.5cm}}
\hline
{\thead{}} & \thead{Khan\\ Academy} & \thead{Academy\\ LMS} & \thead{Axonify} & \thead{Matrix \\LM} & 
\thead{Talent \\ LMS} & 
\thead{Moodle \\+ Level\\ Up!} &
\thead{Turma\\ de \\elite} \\ \hline
    Código aberto               &   &   &   &   & X & X & X               \\ \hline
    Customização de atividades  &   & X & X & X & X & X & X               \\\hline
    Medalhas                    & X & X &   & X &   & X & X               \\ \hline
    Tiers / ligas               &   &   &   &   &   &   & X               \\ \hline
    Leaderboards                & X & X & X & X & X & X & X               \\ \hline
    Visão do Gestor             &   &   &   & X &   &   & X \\ \hline   \end{tabular}
\legend{Fonte: Os autores}

\end{table}
\vspace{600mm}