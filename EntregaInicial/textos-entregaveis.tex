\section{Entregáveis}
A entrega do projeto é dividida em três fases: a prova de conceito, o mínimo produto viável e o produto pronto.

\subsection{Prova de Conceito (POC)}
A \ac{poc} teve como principal objetivo a validação dos elementos de arquitetura apresentados. Então, para isso, foram desenvolvidas as funcionalidades:
 \begin{itemize}
     \item Cadastro e acesso à aplicação pelo administrador.
 \end{itemize}

\subsection{Mínimo Produto Viável (MVP)}
Para o \ac{mvp}, foram incluídas as funcionalidades:
\begin{itemize}
    \item Perfis do aluno, do professor, do gestor e do administrador;
    \item \ac{crud} completo dos usurários, turmas, conquistas, atividades e escolas;
    \item Parametrização de turmas;
    \item Painéis de turmas e de conquistas;
    \item Ranking de alunos.
\end{itemize}

\subsection{Produto Pronto}
Por fim, para o produto pronto, serão consideradas as funcionalidades:

\begin{itemize}
    \item \glspl{dashboard};
    \item Integração com o Moodle.
\end{itemize}

Além de tornar o \textit{design} da aplicação responsivo.