% ---
% RESUMOS
% ---
% resumo em português
\setlength{\absparsep}{18pt} % ajusta o espaçamento dos parágrafos do resumo
\begin{resumo}

 \vspace{\onelineskip}

 A gamificação consiste no uso de mecânicas e características de jogos em atividades que, inicialmente, não aplicam os elementos dos jogos. O principal objetivo desse trabalho é apresentar a aplicação Turma de Elite, um sistema web de gestão de aprendizado que implementa conceitos de gamificação e possui como principal público-alvo os estudantes das redes de ensino, contemplando, em primeira instância, os alunos do 6$^\circ$ ao 9$^\circ$ ano do ensino fundamental, para depois abranger os outros níveis do ambiente escolar, inclusive visando o ensino superior. Serão abordados neste documento questões como a arquitetura e escopo do projeto, tecnologias utilizadas na aplicação e outros pontos relacionados ao sistema, bem como todo processo de produção do mesmo, considerando levantamentos e descartes efetuados. 
 Propõe-se, desse modo, criar uma alternativa tecnológica que promova um maior engajamento dos estudantes no processo de aprendizado, de modo que eles tenham uma maior motivação para adquirir conhecimento por meio dos estudos. Sob essa perspectiva, a gamificação pode ser considerada uma poderosa ferramenta de engajamento, sendo que o desenvolvimento de sistemas que apliquem tais conceitos é totalmente possível.
 
 Para o desenvolvimento da aplicação foi utilizado Angular para o front-end, Firebase authentication para a autenticação dos dados do usuário, servidor SMTP da Google para envio de e-mails de confirmação, o serviço S3 da Amazon para armazenar arquivos enviados pelos usuários, PostgreSQL como banco de dados, Spring Boot para criação e configuração da aplicação, Spring Data para auxiliar no acesso aos dados do banco de dados e Heroku para a hospedagem dos serviços.
 
 \vspace{\onelineskip}
 
 \textbf{Palavras-chave}: Gamificação. Sistema de gestão de aprendizado. Engajamento nos estudos.
 
\end{resumo}

% resumo em inglês
\begin{resumo}[Abstract]
 \begin{otherlanguage*}{english}
 
    \vspace{\onelineskip}
 
     Gamification consists in the use of mechanics and games features in activities that does not apply the game concepts initially. The main purpose of this paper is present the Turma de Elite application, a web Learning Management System that implements gamification concepts and have the education network students as the main target audience, contemplating, in the first instance, 6th to 9th elementary school students, to then cover the other education levels, including aiming the university education.
     This document will address subjects such as architecture, project scope, technologies used and other points related to the system, as well as the entire production process, considering surveys and disposals that were made.
     Thus, it is proposed to create a technological alternative that leads to greater engagement of the students in the learning process, so that they would be more motivated to acquire knowledge through their studies. From this perspective, it is proven that gamification is a powerful engagement tool, being that the development of systems that applies this concepts is totally possible.
     
    \todo[inline]{fazer tradução do resumo, não utilizar tradução automática}
   \vspace{\onelineskip}
   \noindent 
   \textbf{Keywords}: Gamification. Learning Management System. Engagement in studies.
 \end{otherlanguage*}
\end{resumo}