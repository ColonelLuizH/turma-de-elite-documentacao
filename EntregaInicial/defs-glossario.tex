
% Definições para glossario

% ATENCAO o SHARELATEX GERA O GLOSSARIO/LISTAS DE SIGLAS SOMENTE UMA VEZ
% CASO SEJA FEITA ALGUMA ALTERAÇÃO NA LISTA DE SIGLAS OU GLOSSARIO É NECESSARIO UTILIZAR A OPÇÃO :
% "Clear Cached Files" DISPONIVEL NA VISUALIZAÇÃO DOS LOGS 
% ---
% https://www.sharelatex.com/learn/Glossaries

%\newglossaryentry{aberdeen}{
%                name={Aberdeen},
%                plural={Aberdeen},
%                description={uma divisão da Spiceworks Ziff Davis, com mais de três décadas de experiência em pesquisas de mercado independentes e confiáveis}    
%}

\newglossaryentry{api-java}{
                name={API Java},
                plural={API Java},
                description={Conjunto de rotinas e padrões de programação desenvolvidos na linguagem Java}
}

\newglossaryentry{app}{
                name={app},
                plural={apps},
                description={Abreviação de "aplicativo", que é um software desenvolvido para dispositivos móveis}    
}

\newglossaryentry{applet}{
                name={applet},
                plural={applets},
                description={Pequenos programas feitos em Java} 
}

\newglossaryentry{back-end}{
                name={back-end},
                plural={back-end},
                description={Forma de desenvolvimento que se relaciona com o que está por trás das aplicações desenvolvidas na programação.}    
}

\newglossaryentry{branch-develop}{
                name={branch develop},
                plural={branch develop},
                description={possui todo código já entregue e as últimas de desenvolvimento para a próxima versão}
}

\newglossaryentry{bootstrap}{
                name={Bootstrap},
                plural={Bootstrap},
                description={Framework web com código-fonte aberto para desenvolvimento de componentes de interface e front-end}
}

\newglossaryentry{build}{
                name={build},
                plural={build},
                description={Construção do projeto após a compilação}
}

\newglossaryentry{checkpoint}{
                name={checkpoint},
                plural={checkpoints},
                description={Ponto de checagem}    
}

\newglossaryentry{cloud-logging}{
                name={Cloud Logging},
                plural={Cloud Logging},
                description={serviço em nuvem que monitora, agrega, indexa e analisa todos os dados de log}  
}

\newglossaryentry{compilador-java}{
                name={compilador Java},
                plural={compiladores Java},
                description={Transforma linguagem de alto nível em algo mais compreensível pela máquina, sendo escrito para a linguagem Java}  
}

\newglossaryentry{container}{
                name={container},
                plural={containers},
                description={Pacote de aplicação que é gerado para facilitar a portabilidade}  
}

\newglossaryentry{dashboard}{
                name={Dashboard},
                plural={Dashboards},
                description={Painel que apresenta um conjunto de informações de maneira gráfica}
}

\newglossaryentry{daily}{
                name={Daily},
                plural={Daily},
                description={Reunião rápida, com duração máxima de 15 minutos, realizada no início de cada dia de desenvolvimento da Sprint}
}

\newglossaryentry{deep-source}{
                name={Deep Source},
                plural={Deep Source},
                description={Ferramenta para revisão de código por meio de análise estática}  
}

\newglossaryentry{deploy}{
                name={deploy},
                plural={deploy},
                description={Implantação do software}  
}

\newglossaryentry{dyno}{
                name={dyno},
                plural={dynos},
                description={contêineres de tempo de execução gerenciada baseados no sistema operacional Linux}
}

\newglossaryentry{eslint-schematics}{
                name={Eslint Schematics},
                plural={Eslint Schematics},
                description={Plugin que contém regras específicas para projetos em Angular}
}

\newglossaryentry{framework}{
                name={framework},
                plural={frameworks},
                description={abstração que une códigos comuns entre vários projetos de software provendo uma funcionalidade genérica amplamente utilizada por desenvolvedores}
}

\newglossaryentry{front-end}{
                name={front-end},
                plural={front-end},
                description={Local onde é encontrado o design de uma aplicação ou site e suas ferramentas de interação com o usuário}
}

\newglossaryentry{git}{
                name={Git},
                plural={Git},
                description={sistema de controle de versão open-source}
}

\newglossaryentry{github}{
                name={GitHub},
                plural={GitHub},
                description={Plataforma de hospedagem de código-fonte e arquivos com controle de versão usando o Git}
}

\newglossaryentry{hardware}{
                name={hardware},
                plural={hardware},
                description={Qualquer equipamento que compõe a parte física dos computadores} 
}

\newglossaryentry{java-runtime-environment}{
                name={JRE},
                plural={JRE},
                description={Ambiente de execução para aplicações desenvolvidas com a lingugagem Java}
}

\newglossaryentry{journalctl}{
                name={Journalctl},
                plural={Journalctl},
                description={Ferramenta utilizada para o gerenciamento de logs}
}

\newglossaryentry{link}{
                name={link},
                plural={links},
                description={Ferramenta utilizada para o gerenciamento de logs}
}

\newglossaryentry{log}{
                name={log},
                plural={logs},
                description={Registro de algum evento ocorrido em um sistema computacional}
}

\newglossaryentry{merge}{
                name={merge},
                plural={merge},
                description={operação que concilia várias alterações feitas a uma coleção de arquivos controlados por versão}
}

\newglossaryentry{notion}{
                name={Notion},
                plural={Notion},
                description={Aplicativo e serviço de notas com suporte a descontos que integra tarefas}
}


\newglossaryentry{node.js}{
                name={Node.js},
                plural={Node.js},
                description={Software de código aberto, multiplataforma, baseado no interpretador V8 do Google e que permite a execução de códigos JavaScript fora de um navegador web}
}

\newglossaryentry{open-source}{
                name={open source},
                plural={open source},
                description={Licença de código permissiva e gratuita}
}

\newglossaryentry{plugin}{
                name={plugin},
                plural={plugin},
                description={Ferramenta que se encaixa a outro programa principal a fim de prover novas funcionalidades}
}

\newglossaryentry{pull-request}{
                name={pull request},
                plural={pull request},
                description={Pedido direcionado ao repositório principal a fim deste captar as mudanças}
}

\newglossaryentry{product-backlog}{
                name={Product Backlog},
                plural={Product Backlog},
                description={Lista referente ao acúmulo de tarefas em um determinado período de tempo}
}

\newglossaryentry{product-owner}{
                name={Product Owner},
                plural={Product Owner},
                description={sistema de controle de versão open-source}
}

\newglossaryentry{qr-code}{
                name={QR-Code},
                plural={QR-Code},
                description={Código de barras, ou barrametrico, bidimensional, que pode ser facilmente escaneado usando a maioria dos telefones celulares equipados com câmera.}
}


\newglossaryentry{query}{
                name={query},
                plural={queries},
                description={Comandos que, ao serem executados, realizam operações no banco de dados, como pesquisa por registros, inserção, deleção e atualização.}
}

\newglossaryentry{ranking}{
                name={ranking},
                plural={rankings},
                description={Classificação ordenada de acordo com determinados critérios}
}

\newglossaryentry{react}{
                name={React},
                plural={React},
                description={Biblioteca JavaScript de código aberto com foco em criar interfaces de usuário em páginas web}
}

\newglossaryentry{react-i18next}{
                name={react-i18next},
                plural={react-i18next},
                description={Framework de internacionalização para React e React Native}
}

\newglossaryentry{scrum-master}{
                name={Scrum Master},
                plural={Scrum Master},
                description={Membro da equipe que é encarregado de coordenar e potencializar o trabalho da equipe através da metodologia ágil Scrum}
}

\newglossaryentry{software}{
                name={software},
                plural={software},
                description={Conjunto de componentes lógicos de um computador}
}

\newglossaryentry{sprint}{
                name={sprint},
                plural={sprints},
                description={Reunião periódica de pessoas envolvidas em um projeto para definição de metas}
}

\newglossaryentry{sun-microsystems}{
                name={Sun Microsystems},
                plural={Sun Microsystems},
                description={Empresa criadora da linguagem Java que foi comprada pela Oracle em 2009}
}

\newglossaryentry{tech-lead}{
                name={Tech Lead},
                plural={Tech Lead},
                description={Líder da equipe de desenvolvedores em um projeto}
}

\newglossaryentry{tier}{
                name={tier},
                plural={tiers},
                description={Termo em inglês que significa "classificação"}
}

\newglossaryentry{vendor-lock-in}{
                name={Vendor Lock-in},
                plural={Vendor Lock-in},
                description={sistema de documentos em hipermídia que são executados na Internet (abreviação de World Wide Web)}
}

\newglossaryentry{web}{
                name={web},
                plural={web},
                description={sistema de controle de versão open-source}
}

\newglossaryentry{wireframe}{
                name={Wireframe},
                plural={Wireframe},
                description={protótipo usado em design de interface para sugerir a estrutura de uma página web}
}