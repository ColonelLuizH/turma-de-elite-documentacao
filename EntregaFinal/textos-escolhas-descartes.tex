\section{Escolhas e Descartes}
No início do projeto, foi decidido que para o produto pronto ficaria apenas a integração com o Moodle. Entretanto, para complementá-la, foi necessário ter novas ideias. Uma das ideias que surgiram foi inserir um sistema de boletim na aplicação, entretanto foi logo descartada, pois o Moodle já realiza essa função. Por fim, foi decidido que também ficaria para a próxima entrega os dashboards, além de uma nova funcionalidade como meio de pagamento e também trazer responsividade para a aplicação.  

Outra mudança que também ocorreu foi a utilização do PostgreSQL, ao invés do MySQL, para o armazenamento de dados. Foi uma recomendação do professor orientador Daniel Marques Gomes de Morais, visto que o PostgreSQL é um sistema gerenciador de banco de dados que tem integração com o Heroku, portanto não gera custos adicionais de hospedagem. Com essa mudança, a utilização do Oracle Cloud também foi descartada, visto que ele era usado apenas para hospedar o MySQL e não possuía integração gratuita com o \ac{sgbd}.
