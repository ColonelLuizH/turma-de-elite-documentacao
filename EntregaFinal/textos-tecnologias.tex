\section{Tecnologias Utilizadas}
As tecnologias adotadas foram escolhidas com base no conhecimento dos integrantes da equipe e o que traria uma melhor qualidade do projeto para o usuário final, em um menor tempo. Por isso, as tecnologias utilizadas são:

\begin{itemize}
\item{\textbf{Spring}}: \textit{\gls{framework}} adotado no \textit{\gls{back-end}}, que fornece um modelo abrangente de programação e configuração para aplicativos empresariais modernos baseados em Java. Foi escolhido devido à familiaridade de alguns dos integrantes da equipe com esse \textit{\gls{framework}} e também devido a agilidade de desenvolvimento de código proporcionado por ele;

\item{\textbf{Angular}}: \textit{\gls{framework}} escolhido para o \textit{\gls{front-end}} do projeto, é uma plataforma de desenvolvimento baseada em componentes para a construção de aplicativos da \textit{\gls{web}} escaláveis. Foi escolhido devido à experiência da maioria dos integrantes da equipe e por possuir bibliotecas que facilitam a criação e estilização das telas;

\item{\textbf{PostgreSQL}}: usado para armazenar os dados da aplicação, foi escolhido por ser um sistema gerenciador de banco de dados objeto-relacional, de código aberto e de fácil hospedagem no Heroku;

\begin{comment}
\item{\textbf{MySQL}}: escolhido para o armazenamento de dados. Esse é o banco de dados de código aberto mais popular do mundo. Com seu desempenho comprovado, confiabilidade e facilidade de uso, o MySQL se tornou a principal escolha de banco de dados para aplicativos baseados na web, usados por propriedades da \textit{\gls{web}} de alto perfil, incluindo Facebook, Twitter, YouTube, Yahoo! dentre outros \cite{mysql:2021}.
\end{comment}

\item{\textbf{GitHub}}: escolhido para o versionamento de código, pois oferece um serviço de hospedagem de repositório \gls{git} baseado em nuvem, tornando assim muito mais fácil para que a equipe utilize-o para controle de versão e colaboração;


\item{\textbf{SVN}}: também adotado para o controle de versão do projeto, por ser o sistema de controle de versão oficial da instituição de ensino e portanto é um dos requisitos para a disciplina;

\item{\textbf{Heroku}}: é uma plataforma em nuvem que permite às empresas criar, entregar, monitorar e dimensionar aplicativos, sendo a maneira mais rápida de ir da ideia à \ac{url}. E foi adotado para hospedar o back-end da aplicação por ser uma aplicação gratuita e devido à experiência dos integrantes da equipe com o uso dessa plataforma;

\item{\textbf{Travis CI}}: utilizado como plataforma de \ac{ci}, foi adotado por ser um serviço de integração no qual é possível sincronizar projetos de código aberto hospedados no GitHub \cite{travis:2021};

\item{\textbf{Firebase Hosting}}: é um serviço usado para hospedar o  \textit{\gls{front-end}} da aplicação e foi escolhido devido a sua facilidade e agilidade em implantar aplicativos da \textit{\gls{web}};

\item{\textbf{Firebase Authentication}}: utilizado para a autenticação da aplicação, pois oferece os serviços, \ac{sdk} e bibliotecas prontas para autenticar usuários. Sabendo-se que a utilização deste serviço pode trazer à aplicação a característica de \textit{\gls{vendor-lock-in}}, isto é, dependente de uma tecnologia terceira, foi utilizado pois oferece vantagens na segurança e na facilidade de implementação;  

\item{\textbf{Amazon S3}}: O Amazon Simple Sotrage Service é um serviço da \ac{aws} útil para fornecer armazenamento e recuperar qualquer volume de dados por meio de uma interface de serviços web. Este serviço foi escolhido pela equipe por sua facilidade de manuseamento, tendo uma interface simples e intuitiva.

\end{itemize}